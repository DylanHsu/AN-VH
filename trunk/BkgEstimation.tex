\subsection{Ad-hoc correction to the spectra of W boson transverse momentum}
In the resolved category \WlnH\ control regions a downward slope in the data/MC ratio is observed for the reconstructed \ptV\ after the full set of corrections and scale factors have been applied.
We believe this to be the result of the interplay of the falling shape of the differential W cross section as a function of boson $\pt$, and 
the use of higher-order W+jets QCD correction factors which are binned very coarsely in LHE-level $H_T$.
We are interested in deriving instead QCD NLO k-factors binned more finely in the boson $\pt$ if time permits.

For now, we use the independent linear re-weighting functions derived in the excellent 2016-only analysis
to correct this slope for \ttbar\ , \Wudscg\ , and the combination of \Wbb\ and single top via a simultaneous fit of the reconstructed \ptV\ in the \WlnH\ control regions to data.
The input PDF for the fit in each control region is a sum of the MC prediction for each process corrected by a linear function of the reconstructed \ptV\ with a slope that is allowed to float in the fit.
The relative composition of the fitted processes in each control region is fixed.
Table \ref{tab:ptWReweighting} lists the fitted slopes and uncertainties for each class of process.

\begin{table}[htbp]
\caption{ Linear correction factors obtained from a simultaneous fit to the \ptV\ distribution in data in the \WlnH\ control regions. }
\label{tab:ptWReweighting}
\begin{center}
\begin{tabular}{c | c c c} 
\hline
Process                       &  \ttbar                 & \Wudscg\                & \Wbb\ + single top \\
\hline
\Fitted Slope (/\GeV)         & 0.000380 $\pm$ 0.000089 & 0.000575 $\pm$ 0.000046 & 0.00167  $\pm$ 0.00013  \\
Norm preserving constant      & 1.064                   & 1.097                   &1.259    \\
\hline
$\Delta$Bkg at 100 GeV (\%)   & $2.6\pm0.89$            & $3.9\pm0.46$            & $9.2\pm3.6$           \\ 
$\Delta$Bkg at 400 GeV (\%)   & $-8.8\pm3.6$            & $-13.3\pm1.8$           & $-40.9\pm5.2$           \\ 
\hline
\end{tabular}
\end{center}
\end{table}

The TOP group has observed similar MC mis-modelling of the reconstructed \ptV\ distribution for 
\ttbar\ events simulated with Powheg [https://twiki.cern.ch/twiki/bin/view/CMS/TopPtReweighting].  
Their prescription for correcting this effect is per-event re-weighting as a function of the 
$\text{p}_T$(top) at generator level. 
%In the \ZnnH\ and \ZllH\ channels the TOP re-weighting is applied to \ttbar\ simulation. 

\subsection{Background normalization}

Instead of using the theoretical cross sections with associated uncertainties to normalize the main analysis backgrounds,
we conservatively assume zero a-priori knowledge of them, and determine them in situ as unconstrained nuisance parameters in the fit.
These so-called ``normalization scale factors'' are correlated across the control regions and signal region,
but uncorrelated between the boosted and resolved categories. This complete sovereignty reflects our ignorance
of the cross sections' $\pt$ dependence, as well as the different reconstruction efficiency for the
\HBB-like system in two fundamentally different approaches.
Moreover, in the \WlnH\ channel, the ad-hoc W $\pt$ correction to account for said $\pt$ dependence 
is not derived or applied in the boosted category.
In principle, this approach is extremely conservative, and some way to partially link these
scale factors could be thought of, but this should be approached carefully.

The above is true for all the backgrounds except for the \Wb\ process in the \WlnH\ boosted category, which
is so miniscule that we instead assign a Gaussian-constrained scale factor of $1 \pm 0.25$.

The following table shows the normalization scale factors for the background processes, as determined (for now) only in the \WlnH\ channel:

\begin{center}
\begin{tabular}{r|rr} \hline\hline
Process & \WlnH\ resolved & \WlnH\ boosted  \\ \hline
\ttbar\ & $0.87 \pm 0.02$ & $0.53 \pm 0.03$ \\
W+LF    & $1.06 \pm 0.05$ & $0.88 \pm 0.04$ \\
\Wb\    & $2.44 \pm 0.13$ & $1.06 \pm 0.21$ \\
\Wbb\   & $1.58 \pm 0.11$ & $1.32 \pm 0.4$  \\
\hline\hline
\end{tabular}
\end{center}

