%\section{Signal normalization}



\subsection{Frequentist Interpretation}

Given the binned likelihood on data $\mathcal
L(\mathrm{data}|\mu,\boldsymbol{\theta})$, we define the profile
likelihood test statistic following the LHC \CLs procedure~\cite{LHCCLs},
\begin{equation}
\tilde q_{\mu} = -2\log\frac{\mathcal
  L(\mathrm{data}|\mu,\boldsymbol{\hat\theta}_{\mu})}{\mathcal
  L(\mathrm{data}|\hat\mu, \boldsymbol{\hat\theta})} ~,~~
0\leq\hat\mu\leq\mu~,
\label{eqn:LHCteststat}
\end{equation}
where $\boldsymbol{\hat\theta}_{\mu}$ refers to the conditional maximum
likelihood estimators of $\boldsymbol{\theta}$ assuming a given value
$\mu$, and $\hat\mu$ and $\boldsymbol{\hat\mu}$ correspond to the
global maximum of the likelihood. In the asymptotic regime, we may
derive the observed 95\% CL upper limit on the signal strength by
computing the value of $\mu$ that satisfies, 
\begin{equation}
\mathrm{CL}_{\mathrm{s}}\equiv
\frac{\mathrm{CL}_{\mathrm{s+b}}}{\mathrm{CL}_{\mathrm{b}}} = 
\frac{1-\Phi(\sqrt{\tilde
    q_{\mu}})}{\Phi(\sqrt{\tilde q_{\mu,\mathrm{A}}} - \sqrt{\tilde
    q_{\mu}} ) } = \alpha ~,
\end{equation}
where $\alpha = 0.05$, $\tilde q_{\mu,\mathrm{A}}$\footnote{Note $\tilde
  q_{\mu,\mathrm{A}} = \frac{\mu^2}{\sigma_{\mathrm{A}}^2}$ where
  $\sigma_{\mathrm{A}}$ is an estimator for the variance of $\mu$.} is the test statistic evaluated on
the Asimov dataset~\cite{Asimov} corresponding exactly to the expected
background and the nominal nuisance parameters (setting all
statistical fluctuations to zero)~\cite{Cowan:2010js,LHCCLs}, and 
$\Phi(x)$ is the cumulative distribution function of the standard
normal distribution. A similar expression is used to derive the median expected 95\% CL upper limit, 
\begin{equation}
\frac{1-\Phi(\sqrt{\tilde q_{\mu,\mathrm{A}}})}{0.5} = \alpha ~.
\end{equation}
%Importantly, for situations with small numbers of events, the
%asymptotic result is known to give biased (over-optimistic) results~\cite{LHCCLs}.

Conversely, in the case of a discovery, one tests $\mu=0$ and measures
the ``local significance'' using a modified test statistic,
\begin{equation}
q_{0} = -2\log\frac{\mathcal
  L(\mathrm{data}|0,\boldsymbol{\hat\theta}_{0})}{\mathcal
  L(\mathrm{data}|\hat\mu, \boldsymbol{\hat\theta})} ~, ~~ \hat\mu\geq 0~.
\end{equation}
The observed local significance is then simply $Z = \sqrt{q_0}$.
 
\subsection{Expected sensitivity using QCD MC}

The first test of the fit procedure is done by using MC only to predict the final expected sensitivity.
We take pass and fail yields for the signals and $\PW$, $\PZ$, and $\ttbar$ backgrounds from simulation.
%We then take the QCD fail yields also from simulation and define the
%pass yield to be the fail region shape times the efficiency of the
%double b-tag discriminator cut.

In the absence of a signal ($\mu=0$), the 95\% CL median expected limit on the
signal strength is $\mu<3.73$ ($\mu<2.77$ including the signal Higgs \pt
corrections described in Sec.~\ref{sec:signalpt}), accounting for
all systematic uncertainties. Conversely, in the presence of a signal
($\mu=1$), the expected significance is $0.52\sigma$ ($0.76\sigma$ including the signal Higgs \pt
corrections). Fig.~\ref{fig:deltaLL_qcd} shows the profile
likelihood test statistic scan in $\mu$ on the asimov dataset derived
from the background-only MC ($\mu=0$) and the
signal$+$background MC ($\mu=1$). Tab.~\ref{tab:expectedLimits}
lists the corresponding ``stat$+$syst'' expected limits in the absence
of a signal ($\mu=0$) as well as the expected significance in the
presence of a signal ($\mu=1$).

\begin{figure}[hbtp]
\centering
\subfigure[$\mu=0$]{\includegraphics[width=0.48\textwidth]{figures/results/qcd/deltaLL_asimov_r0p000000.pdf}}
\subfigure[$\mu=1$]{\includegraphics[width=0.48\textwidth]{figures/results/qcd/deltaLL_asimov_r1p000000.pdf}}
	\caption{Profile Likelihood test statistic
          $-2\Delta\log(\mathcal{L})$ scan as a function of the signal
          strength $\mu$ on the asimov dataset. The solid black curve,
          labeled ``stat$+$syst,'' corresponds to the scan when
          profiling all nuisance parameters and the dotted blue curve,
          labeled ``stat only,'' corresponds to only profiling the
          unconstrained nuisance parameters: QCD transfer factor
          parameters and the $\ttbar$ scale factors. The
          expected 95\% CL upper limit on $\mu$ with or without the
          systematic uncertaintines corresponds to the $\mu$ value
          for which the corresponding test statistic curve intersects
          $-2\Delta\log(\mathcal{L}) = 3.84$}
	\label{fig:deltaLL_qcd}}
 \end{figure}

\begin{table}[htbp]
\centering
\begin{tabular}{ll}
\hline\hline
Without Higgs \pt corrections, $\mu=0$ & \\
\hline\hline
Expected  2.5\%& $\mu < 2.0204$\\
Expected 16.0\%& $\mu <2.6899$\\
Expected 50.0\%& $\mu< 3.7344$\\
Expected 84.0\%& $\mu< 5.2378$\\
Expected 97.5\%& $\mu< 7.0301$\\
\hline
 $\mu=1$ & \\
\hline
Expected significance & $0.52\sigma$\\
\hline\hline
With Higgs \pt corrections, $\mu=0$ & \\
\hline\hline
Expected  2.5\%& $\mu< 1.4192$\\
Expected 16.0\%& $\mu< 1.9218$\\
Expected 50.0\%& $\mu< 2.7734$\\
Expected 84.0\%& $\mu< 4.1111$\\
Expected 97.5\%& $\mu < 6.0191$\\
\hline
 $\mu=1$ & \\
\hline
Expected significance & $0.76\sigma$\\
\hline\hline

\end{tabular}
\caption{``Stat$+$syst'' expected limits for $35.9\fbinv$ from
  \texttt{combine} in the absense of a signal ($\mu=0$) and expected
  significance in the presense of a signal ($\mu=1$) including the
  the full set of systematic uncertainties and with and without the Higgs \pt corrections.
\label{tab:expectedLimits}}
\end{table}
\clearpage
\subsection{Fit closure}
We perform the closure test for the Higgs signal considering the full set of systematic unceratinty and for $\mu=1$. 
From the observed fits, we do not observe any significant deviation and pulls are found to be consistent with unity, see Fig.~\ref{fig:pullsdataH}.

\begin{figure}[hbtp]
\centering
\includegraphics[width=0.49\textwidth]{figures/Hpulls.png}
 \caption{The pull distribution from generating off the best fit distribution and fitting with the default signal extraction method, for the H signal.}

 \label{fig:pullsdataH}
 \end{figure}

\clearpage

\subsection{Observed results}

In Figs.~\ref{fig:fitdata0}-\ref{fig:fitdata2}, the post-fit distributions of $\mSD$ in data
are shown for the passing and failing categories and compared to the
background prediction for the full dataset. Finally,
Fig.~\ref{fig:qcdTF} shows the two-dimensional QCD transfer factor in
($\mSD,\pt$) and ($\rho,\pt$)  space.


\begin{figure}[hbtp]
\centering
    \includegraphics[width=0.49\textwidth]{figures/results_approval/mlfit_fail_allcats_fit_s.pdf}
    \includegraphics[width=0.49\textwidth]{figures/results_approval/mlfit_pass_allcats_fit_s.pdf}\\
 \caption{Post-fit $m_{\mathrm{SD}}$ distributions in data events for the
      pass and fail regions by summing all the \pt categories and using a
      polynomial 2nd order in $\rho$ and 1st order in $\pt$. The EWK and
      signal events are also shown separately.  }
 \label{fig:fitdata2}
 \end{figure}

\clearpage

\begin{figure}[hbtp]
\centering
    \includegraphics[width=0.49\textwidth]{figures/results_approval/mlfit_fail_cat1_fit_s.pdf}
    \includegraphics[width=0.49\textwidth]{figures/results_approval/mlfit_pass_cat1_fit_s.pdf}\\
    \includegraphics[width=0.49\textwidth]{figures/results_approval/mlfit_fail_cat2_fit_s.pdf}
    \includegraphics[width=0.49\textwidth]{figures/results_approval/mlfit_pass_cat2_fit_s.pdf}\\
    \includegraphics[width=0.49\textwidth]{figures/results_approval/mlfit_fail_cat3_fit_s.pdf}
    \includegraphics[width=0.49\textwidth]{figures/results_approval/mlfit_pass_cat3_fit_s.pdf}\\
 \caption{Post-fit $m_{\mathrm{SD}}$ distributions in data events for the
      pass and fail regions and the first three \pt categories (450-500,500-550,550-600 from top to bottom) by using a
      polynomial 2nd order in $\rho$ and 1st order in $\pt$. 
     The EWK and
      signal events are also shown separately.  }
 \label{fig:fitdata0}
 \end{figure}

\begin{figure}[hbtp]
\centering
    \includegraphics[width=0.49\textwidth]{figures/results_approval/mlfit_fail_cat4_fit_s.pdf}
    \includegraphics[width=0.49\textwidth]{figures/results_approval/mlfit_pass_cat4_fit_s.pdf}\\
    \includegraphics[width=0.49\textwidth]{figures/results_approval/mlfit_fail_cat5_fit_s.pdf}
    \includegraphics[width=0.49\textwidth]{figures/results_approval/mlfit_pass_cat5_fit_s.pdf}\\
    \includegraphics[width=0.49\textwidth]{figures/results_approval/mlfit_fail_cat6_fit_s.pdf}
    \includegraphics[width=0.49\textwidth]{figures/results_approval/mlfit_pass_cat6_fit_s.pdf}\\
 \caption{Post-fit $m_{\mathrm{SD}}$ distributions in data events for the
      pass and fail regions and the last three \pt categories (600-675,675-800,800-1000 from top to bottom) by using a
      polynomial 2nd order in $\rho$ and 1st order in $\pt$. The EWK and
      signal events are also shown separately.  }
 \label{fig:fitdata1}
 \end{figure}


\begin{figure}[hbtp]
\centering
    \includegraphics[width=0.49\textwidth]{figures/results_approval/tf_msdcolz.pdf}
    \includegraphics[width=0.49\textwidth]{figures/results_approval/tf_rhocolz.pdf}
    \caption{Post-fit QCD transfer factor map in ($\mSD,\pt$) (left)
      and ($\rho,\pt$) (right) space.  }
 \label{fig:qcdTF}
 \end{figure}

\clearpage


The measured $\PZ$ boson signal strength is $\mu_\PZ =\muZVal_{-\muZErrLo}^{+\muZErrHi}$,  which corresponds to an observed significance of $\muZObsSig\sigma$ with $\muZExpSig\sigma$ expected. This constitutes the first observation of the $\PZ$ signal in the single-jet topology, further validating the substructure and $\cPqb$-tagging strategy for the Higgs boson search in the same topology. The measured cross section of the $\PZ+$jets process is $\xsecZVal_{-\xsecZErrLo}^{+\xsecZErrHi}\unit{fb}$, which is consistent with the SM expectation within the uncertainty. The measured $\PH$ boson signal strength is $\mu=\muVal_{-\muErrLo}^{+\muErrHi}$, including the $\pt$ corrections described in Sec.~\ref{sec:signalpt}.  The observed (expected) significance is $\muObsSig\sigma$ ($\muExpSig\sigma$). The measured cross section of the $\PH(\bbbar)$ production for $\pt>450\GeV$ is $\xsecVal_{-\xsecErrLo}^{+\xsecErrHi}\unit{fb}$, which is consistent with the SM expectation within the uncertainty.


Tab.~\ref{tab:ObservedSig} lists the corresponding observed
significance and Fig.~\ref{fig:deltaLL}-\ref{fig:deltaLLZ} shows the profile likelihood test
statistic scan as a function of the Higgs and Z boson signal strength parameters
($\mu,\mu_{Z}$) when fitting the data and the post-fit asimov dataset.

Fig.~\ref{fig:fitch2} show the fitted signal strenght per each $\pt$ category for the $\PH$ and $\PZ$ signals in the case the other POI is free to float.


\begin{figure}[hbtp]
\centering
\includegraphics[width=0.48\textwidth]{figures/results_approval/deltaLL_data_r1p000000_floatOtherPOIs.pdf}
\includegraphics[width=0.48\textwidth]{figures/results_approval/deltaLL_data_r_z1p000000_floatOtherPOIs.pdf}
\includegraphics[width=0.65\textwidth]{figures/results_approval/deltaLL2D_data.pdf}
        \caption{Profile likelihood test statistic $-2\Delta\log\mathcal{L}$ scan as a function of the $\PH$ signal
          strength $\mu$ (upper left), $\PZ$ signal strength $\mu_\PZ$ (upper right), and both signal strengths $(\mu,\mu_\PZ)$ (lower).}
        \label{fig:deltaLL}}
 \end{figure}


\begin{figure}[hbtp]
\centering
\includegraphics[width=0.65\textwidth]{figures/results_approval/deltaLL2D_asimov.pdf}
        \caption{Profile likelihood test statistic $-2\Delta\log\mathcal{L}$ scan as a function of the $\PH$ and $\PZ$ signal strength $(\mu,\mu_\PZ)$ when fitting the asimov dataset.}
        \label{fig:deltaLLZ}}
 \end{figure}



\begin{table}[htbp]
\centering
\begin{tabular}{llll}
\hline\hline
 & $\PH$ & $\PH$ no \pt corrections &$\PZ$\\
\hline\hline
Observed best fit   & $\mu=\muVal_{-\muErrLo}^{+\muErrHi}$ & $\mu'=\muNoVal_{-\muNoErrLo}^{+\muNoErrHi}$  & $\mu_\PZ =\muZVal_{-\muZErrLo}^{+\muZErrHi}$\\
%\hline
%Observed limit& $\mu < 6.87$\\
%Expected  2.5\%& $\mu < 2.02$\\
%Expected 16.0\%& $\mu< 2.70$\\
%Expected 50.0\%& $\mu< 3.76$\\
%Expected 84.0\%& $\mu< 5.28$\\
%Expected 97.5\%& $\mu< 7.14$\\
%\hline
Expected significance &  $\muExpSig\sigma$ ($\mu=1$) &$\muNoExpSig\sigma$ ($\mu'=1$) &  $\muZExpSig\sigma$ ($\mu_{\PZ}=1$) \\
Observed significance &  $\muObsSig\sigma$ &$\muNoObsSig\sigma$ &  $\muZObsSig\sigma$\\
\hline\hline
\end{tabular}
\caption{Fitted signal strength and observed significance of the Higgs and $\PZ$ signals.
\label{tab:ObservedSig}}
\end{table}



\begin{figure}[hbtp]
\centering
\subfigure[$\PH$, $\PZ$ floating]{\includegraphics[width=0.49\textwidth]{figures/results_approval/ccc_r.pdf}}
\subfigure[$\PZ$, $\PH$ floating]{\includegraphics[width=0.49\textwidth]{figures//results_approval/ccc_r_z.pdf}}
 \caption{Fitted signal strenghts for the Higgs and Z boson signals for each \pt category by floating the Z and H signal strenght parameter respectively.}
        \label{fig:fitch2}
 \end{figure}

\clearpage
\subsection{Pre- and post-fit systematic uncertainties}


Fig.~\ref{fig:impacts} shows the pulls of the constrained nuisance parameters as well as the impacts on the signal
strength when fitting the data including the Higgs $\pt$ corrections.

\begin{figure}[h!]
\centering
\includegraphics[width=0.72\textwidth]{figures/results_approval/impacts_data_r.pdf}
	\caption{Impacts for nuisance parameters (excluding MC
          statistic uncertainties and freely floating QCD parameters) when fitting the data with Higgs $\pt$ corrections.
	\label{fig:impacts}}
 \end{figure}

Similarly, Fig.~\ref{fig:impactsnoHpt} shows the same pulls excluding
the dominant uncertainty on the signal coming from the theory uncertainty on the gluon fusion cross section at high \pt. 

\begin{figure}[h!]
\centering
\includegraphics[width=0.72\textwidth]{figures/results_approval/impacts_data_r_nothy.pdf}
        \caption{Impacts for nuisance parameters (excluding MC
          statistic uncertainties and freely floating QCD parameters) when fitting the data with Higgs $\pt$ corrections without including the theory uncertainty.
        \label{fig:impactsnoHpt}}
 \end{figure}

The pulls of the constrained nuisance parameters when fitting the data to extract the Z signal are rerpoted in Fig.~\ref{fig:impactsZ}. They are compatible to the Higgs signal extraction case (Fig.~\ref{fig:impactsnoHpt}

\begin{figure}[h!]
\centering
\includegraphics[width=0.72\textwidth]{figures/results_approval/impacts_data.pdf}
        \caption{Impacts for nuisance parameters (excluding MC
          statistic uncertainties and freely floating QCD parameters) when fitting the data to extract the Z signal. 
        \label{fig:impactsZ}}
 \end{figure}

\clearpage
